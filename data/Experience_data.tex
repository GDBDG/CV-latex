\addexperience{
    \cvevent
        {Fullstack developper}
        {EDF, DIPNN via Sogeti}
        {septembre 2022 - }
        {Lyon, France}
        {
            \small{
                \textbf{Technologies} 
            
                \hspace{1.4em} \python\ \BB{} \django\ \BB{} \pyqt\ \BB{} \gitlab\ \BB{} \JS\ \BB{} AgGrid \BB{} Plotly \\
                \textbf{Projet} (Scrum)
                \begin{itemize}
                    \item [\BB] Applications de gestion de crise, domaine nucléaire (bureau avec PyQt, et web avec Django).
                    \item [\BB] Application web : centralise et stocke les données des centrales, interface d'analyse pour la crise, et API pour les données.
                    \item [\BB] Application de bureau : interface permettant de piloter et d'utiliser un coeur de calcul simulant une centrale.
                    \item [\BB] Configuration intégration continue avec pipelines gitlab.
                    \item [\BB] Programmation parallèle et asynchrone.
                \end{itemize}
            }
        }
        {EDF.png}         
}

\addexperience{
    \divider
    \cvevent
        {Backend developper}
        {Sogeti}
        {octobre 2021 - septembre 2022}
        {Lyon, France}
        {
            \small{
                \textbf{Technologies} 
            
                \hspace{1.4em} \python\ \BB{} \flask\ \BB{} \tensorflow\ \BB{} \elk\ \BB{} \docker\ \BB{} \azuredevops\\
                \textbf{Projet} (Scrum)
                \begin{itemize}
                    \item [\BB] Application de détection de voiture dans des images (avec marque, modèle, ...)
                    \item [\BB] Mise en place d'une API de prédiction pour l'application 
                    \item [\BB] Mise en place des scripts d'entrainement, via VM distante (lancement avec pipeline AzureDevops)
                    \item [\BB] Modification des pipelines CI/CD
                \end{itemize}
            }
        }
        {sogeti.png}         
}
\addexperience{
    \divider
    \cvevent
        {PFE (Projet Fin d'Etude)}
        {EDF via Sogeti}
        {Mars - Août 2021}
        {Lyon, France}
        {\small{
            \textbf{Technologies} 

            \hspace{1.4em} \python\ \BB{} \django\ \BB\ \pyqt\ \BB\ \gitlab\ \\
            \textbf{Projet} (Scrum)
            \begin{itemize}
                \item [\BB] Applications de gestion de crise, domaine nucléaire (bureau avec PyQt, et web avec Django)
                \item [\BB] Récupération de statistiques d'usage des fonctionnalités des applications 
            \end{itemize}
        }}
        {EDF.png}
}

\addexperience{
    \divider
    \cvevent
        {Stage AI}
        {Serpico}
        {Juin - Août 2021}
        {Luxembourg}
        {
            \small{
                \textbf{Technologies} 

                \hspace{1.4em} \python\ \BB{} \flask\ \BB{} \php\ \BB{} symfony \BB{} \docker\ \\
                \textbf{Projet}
                \begin{itemize}
                    \item [\BB] Mise en place d'un micro service de calcul statistique
                    \item [\BB] Mise à jour d'une appli web (de silex vers symfony)
                \end{itemize}
            }
        }
        {serpico.png}        
}